% Paquetes de generalidades
%%%%%%%%%%%%%%%%%%%%%%%%%%%%%%%

% Para escribir tildes y eñes
\usepackage[utf8]{inputenc}   

% Para que los títulos de figuras, tablas y otros estén en español
\usepackage[spanish,es-noquoting]{babel} 
	% Cambiar nombre a tablas
	\addto\captionsspanish{\renewcommand{\tablename}{Tabla}}	
    % Cambiar nombre a lista de tablas		
	\addto\captionsspanish{\renewcommand{\listtablename}{Índice de tablas}}	
    % Cambiar nombre a capítulos
	\addto\captionsspanish{\renewcommand{\chaptername}{Sección}}

% Para que la bibligrafía esté en español

\decimalpoint
\usepackage[spanish,es-tabla]{babel}

\usepackage[backend=bibtex]{biblatex}
\addbibresource{referencias.bib}


% Tamaño del área de escritura de la página	
\usepackage{geometry}                         
	\geometry{left=18mm,right=18mm,top=23mm,bottom=23mm} 	

% Paquetes para matemática
%%%%%%%%%%%%%%%%%%%%%%%%%%%%%%%

% Los paquetes ams son desarrollados por la American Mathematical Society y mejoran la escritura de fórmulas y símbolos matemáticos.
\usepackage{amsmath}       
\usepackage{amsfonts}     	
\usepackage{amssymb}

% Paquetes para manejo de gráficas y figuras
%%%%%%%%%%%%%%%%%%%%%%%%%%%%%%%

% Para insertar gráficas
\usepackage{graphicx}     	

% Para colocar varias subfiguras
\usepackage[lofdepth,lotdepth]{subfig}

% Para crear gráficos vectoriales con un lenguaje descriptivo/geométrico
\usepackage{tikz}

% Para crear circuitos vectoriales basados en TikZ
\usepackage[american]{circuitikz}

% Paquetes relacionados con el estilo 
%%%%%%%%%%%%%%%%%%%%%%%%%%%%%%%

% Para la presentación correcta de magnitudes y unidades
\usepackage{siunitx}	

% Para hipervínculos y marcadores
\usepackage[colorlinks=true,urlcolor=blue,linkcolor=blue,citecolor=green]{hyperref}
	\urlstyle{same}

% Para ubicar las tablas y figuras justo después del texto
\usepackage{float}	

% Para hacer tablas más estilizadas
\usepackage{booktabs}		

% Para hacer secciones con múltiples columnas
\usepackage{multicol}

% Para insertar código fuente estilizado
\usepackage{listings}
	\lstset{basicstyle=\ttfamily,breaklines=true}
    \lstset{numbers=left, numberstyle=\tiny, stepnumber=1, numbersep=6pt}

% Para agregar código con formato de Matlab
%\usepackage[numbered,autolinebreaks]{mcode}

% Para utilizar el número de páginas
\usepackage{lastpage}

% Para manejar los encabezados y pies de página
\usepackage{fancyhdr}
	% Contenido de los encabezados y pies de pagina
	\pagestyle{fancy}

% Misceláneos
%%%%%%%%%%%%%%%%%%%%%%%%%%%%%%%

% Para insertar símbolos extraños
\usepackage{marvosym}

%%libreria tikz

\usepackage{tikz}
\usetikzlibrary{shapes.geometric, arrows}