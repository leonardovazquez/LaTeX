Simplificar implica lograr que una expresión matemática esté compuesta por un único término lo más sencillo posible. Veamos un ejemplo:

\begin{equation}
    \notag
    \frac{0.\wideparen{2}+1,1\wideparen{1}}{0,\wideparen{6}} -1 =?
\end{equation}

Desde el capitulo anterior, sabemos que a los números decimales podemos expresarlos como números fraccionarios:

\begin{equation}
    \notag
    \frac{\frac{2}{9}+\frac{100}{90}}{\frac{6}{9}}-1 =?
\end{equation}

Lo próximo a realizar es una suma de fracciones con distinto denominador: 

\begin{equation}
    \notag
    \frac{\frac{4}{3}}{\frac{6}{9}}-1 = ?
\end{equation}

División de fracciones y luego una simple resta:

\begin{equation}
    \notag
    2-1 = 1
\end{equation}

A pesar de que el proceso sea largo, lo importante aquí a resaltar es la aplicación de los conceptos y propiedades vistas anteriormente.

Por otro lado, racionalizar implica modificar una expresión fraccionaria con raíz en el denominador a una expresión fraccionaria sin raíz en el denominador. El truco es bastante sencillo: Hay que observar el índice de la raíz y la potencia del radicando:

\begin{equation}
    \notag
    \frac{1}{\sqrt[n]{a^b}}
\end{equation}

Luego se multiplica por un entero cuyas raíces en numerador y denominador son iguales y posee igual índice pero distinta potencia al que se desea extraer o racionalizar:

\begin{equation}
    \notag
    \frac{1}{\sqrt[n]{a^b}}\cdot \frac{\sqrt[n]{a^{n-b}}}{\sqrt[n]{a^{n-b}}}
\end{equation}

Haciendo uso de alguna de las propiedades de la potencia y raíz:


\begin{equation}
    \notag
   \sqrt[n]{a^{n-b}} = a^{(n-b)/n}
\end{equation}

\begin{equation}
    \notag
   \sqrt[n]{a^b} = a^{b/n}
\end{equation}

De esta forma: 

\begin{equation}
    \notag
    \frac{1\cdot \sqrt[n]{a^{n-b}} }{a^{b/n} \cdot  a^{(n-b)/n}} 
\end{equation}

Pero exponentes de potencias con igual base se suman por lo que:

\begin{equation}
    \notag
   a^{b/n} \cdot  a^{(n-b)/n}= a 
\end{equation}

Por lo que la formula para racionalizar es simplemente: 

\begin{equation}
    \notag
    \frac{1}{\sqrt[n]{a^b}} = \frac{\sqrt[n]{a^{n-b}}}{a}
\end{equation}

Veamos un ejemplo sencillo:


\begin{equation}
    \notag
    \frac{4}{\sqrt[5]{7^3}} 
\end{equation}

En este caso solo basta con mirar nuestra formula y darse cuenta que: $n=5$, $a=7$ y $b=3$ por lo que $n-b=2$. De esta forma:

\begin{equation}
    \notag
    \frac{4}{\sqrt[5]{7^3}} = \frac{4\cdot \sqrt[5]{7^{2}}}{7}
\end{equation}

En muchos casos, puede que te encuentres con expresiones del tipo:

\begin{equation}
    \notag
    \frac{1}{\sqrt{a}+\sqrt{b}}
\end{equation}

La manera de racionalizar esto es multiplicando por un entero que posea en el numerador y denominador el conjugado del denominador de la expresión anterior, es decir, cambiar el signo:

\begin{equation}
    \notag
    \frac{1}{\sqrt{a}+\sqrt{b}}\cdot \frac{\sqrt{a}-\sqrt{b}}{\sqrt{a}-\sqrt{b}}
\end{equation}

Realizando la multiplicación entre denominadores se obtiene fácilmente:

\begin{equation}
    \notag
     \frac{1}{\sqrt{a}+\sqrt{b}}= \frac{\sqrt{a}-\sqrt{b}}{a-b}
\end{equation}
$Observación$: En caso de que en el denominador no sean ambos términos con raíz necesariamente, el procedimiento consiste en multiplicar por un entero conjugado.\\
\medskip
Veamos un ejemplo sencillo: 

\begin{equation}
    \notag
    \frac{1}{\sqrt{2}+\sqrt{3}} 
\end{equation}
Multiplicamos por el entero conjugado:

\begin{equation}
    \notag
    =\frac{1}{(\sqrt{2}+\sqrt{3})} \cdot \frac{\sqrt{2}-\sqrt{3}}{(\sqrt{2}-\sqrt{3})}
\end{equation}

\begin{equation}
    \notag
=\frac{\sqrt{2}-\sqrt{3}}{(\sqrt{2}\cdot \sqrt{2})-(\sqrt{3}\cdot \sqrt{2})+(\sqrt{3}\cdot \sqrt{2})-(\sqrt{3}\cdot \sqrt{3})}
\end{equation}

Finalmente:
\begin{equation}
    \notag
    \frac{1}{\sqrt{2}+\sqrt{3}} = \frac{\sqrt{2}-\sqrt{3}}{2-3} =  \frac{\sqrt{2}-\sqrt{3}}{-1} 
\end{equation}

Ejercicios: Simplificar

\begin{enumerate}
\renewcommand{\labelenumi}{{\theenumi})}
\item $7\cdot \sqrt[3]{9}- \frac{1}{3}\cdot  \sqrt[3]{9}$
\item $\frac{3}{5}\cdot \sqrt{7}-\frac{3}{4}$
\item $3\cdot \sqrt[3]{64}-\sqrt[3]{27}$
\item $\sqrt{80}\cdot \sqrt{30}$
\end{enumerate}

Ejercicios: Resolver

\begin{enumerate}
\renewcommand{\labelenumi}{{\theenumi})}
\item $(0,1)^{-1}+\frac{1}{2}:\frac{-3}{2}-\sqrt{1/4}$
\item $0,3\cdot \frac{-1}{2}-(2/5)^{-2}$
\item $(\frac{3}{2}-\frac{5}{3})^{-1}+\sqrt{0,\wideparen{1}}$
\end{enumerate}

Ejercicios: Racionalizar

\begin{enumerate}
\renewcommand{\labelenumi}{{\theenumi})}
\item ${13}:(20-\sqrt{15})$
\item $-17:\sqrt{15}$
\item $8:\sqrt[10]{2^4}$
\end{enumerate}