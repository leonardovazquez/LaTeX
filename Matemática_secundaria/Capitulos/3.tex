Las potencias de un número son operaciones que hacen multiplicar dicho número por sí mismo. Dicho número de multiplicaciones se le llama $Potencia$ o $exponente$. El número a multiplicar se le dice $base$. Veamos un ejemplo:

\begin{equation}
    \notag
    5^2=5\cdot5=25
\end{equation}


El cual es $5$ (base) $al$ $cuadrado$ (exponente $2$). Si elevamos $5$ $a$ $la$ $potencia$ $de$ $3$, es decir $(5^3)$, se le dice $al$ $cubo$. Si elevamos $5$ $a$ $la$ $potencia$ $de$ $4$, el cual es $(5^4)$, se le dice $a$ $la$ $cuarta$ y así sucesivamente.\\
\medskip

Veamos algunas de las propiedades básicas: \\

\begin{itemize}
    \item Potencia neutra: 
    \begin{equation}
        \notag
        a^1=a \longrightarrow 2^1=2
    \end{equation}
    \item Potencia de exponente nulo es una unidad: 
    \begin{equation}
        \notag
        a^0=1 \longrightarrow 2^0=1
    \end{equation} 
    \item Producto de igual base se suman exponentes: 
    \begin{equation}
        \notag
        a^n \cdot a^m = a^{n+m} \longrightarrow 2^2\cdot 2^1=2^3 
    \end{equation} 
    \item Exponente negativa implica división: 
    \begin{equation}
        \notag
        a^{-n}=1/a^n \longrightarrow 2^{-1}=1/2  
    \end{equation} 
    \item División de potencias de igual base es resta de exponentes: 
     \begin{equation}
        \notag
        a^m / a^n = a^{m-n} \longrightarrow 2^3/2^1 = 2^2
    \end{equation} 
    \item Potencia de potencia se multiplican exponentes:
     \begin{equation}
        \notag
        {(a^n)}^m = a^{n\cdot m} \longrightarrow {(2^2)}^2 = 2^4 
    \end{equation}  
    \item Potencia de distintas bases:
     \begin{equation}
        \notag
        (a\cdot b)^n =a^n\cdot b^n \longrightarrow {(2\cdot 3)}^2 = 2^2 \cdot 2^3
    \end{equation}  
    \item Potencia de división:
     \begin{equation}
        \notag
        (a/ b)^n =a^n / b^n \longrightarrow {(2 / 3)}^2 = 2^2 / 2^3
    \end{equation}  
\end{itemize}

¡Son muchas propiedades! Tranquilo/a, puede que no las utilices todas.\\
\medskip

Por otro lado, la Raíz es la operación inversa a la Potencia. La raíz de un número $a$ es un número $b$ que multiplicado tantas veces por sí mismo da el número $a$. Veamos un ejemplo de la raíz cuadrada (El número se denomina índice y el número dentro de la raíz se llama radicando): \\


\begin{equation}
\notag
\sqrt[2]{36}=\sqrt{36}=6 \longrightarrow 6\cdot 6 = 36 
\end{equation} 

Y de la raíz cúbica: \\
\begin{equation}
\notag
\sqrt[3]{125}=5 \longrightarrow 5\cdot 5 \cdot 5 = 125 
\end{equation}


Como era de anticiparse, existe una relación entre la raíz de tal número con la potencia:\\

\begin{equation}
\notag
\sqrt[n]{a^m}=a^{m/n} \longrightarrow \sqrt[3]{2^3}=2^{3/3}=2     
\end{equation}



Veamos algunas de las propiedades básicas: \\

\begin{itemize}
    \item Raíz de producto: 
    \begin{equation}
        \notag
        \sqrt[n]{a\cdot b} = \sqrt[n]{a} \cdot \sqrt[n]{b} \longrightarrow \sqrt{2\cdot 3} = \sqrt{2} \cdot \sqrt{3}
    \end{equation}
    
    \item Raíz de división: 
    \begin{equation}
        \notag
        \sqrt[n]{a: b} = \sqrt[n]{a} : \sqrt[n]{b} \longrightarrow \sqrt{2: 3} = \sqrt{2} : \sqrt{3}
    \end{equation}
    
    \item Como caso especial:
      \begin{equation}
        \notag
         \sqrt[2]{a}\cdot \sqrt[2]{a}=\sqrt[2]{a}^2 = a     
            \end{equation}

\end{itemize}

\medskip
Ejercicios:

\begin{enumerate}
\renewcommand{\labelenumi}{{\theenumi})}
\item $8^2$
\item $3^3$
\item $\sqrt{9}$
\item $\sqrt{100}$

\end{enumerate}