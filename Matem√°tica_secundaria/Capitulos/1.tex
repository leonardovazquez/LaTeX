Los números naturales son aquellos números exactos y positivos: 1, 2, 3, 4, 5... Y así sucesivamente. \\ Hay varias propiedades para una de las operaciones básicas (suma, resta, multiplicación y división):
\begin{itemize}
\item Propiedad conmutativa (suma): \vskip \centering $5 + 8 = 13$
                             \vskip \centering $8 + 5 = 13$ \vskip
\raggedright 
\item Propiedad conmutativa (multiplicación): \vskip \centering $2\cdot3 =6$
                             \vskip \centering $3\cdot2= 6$
\vskip
\raggedright 
\item Propiedad asociativa (suma): \vskip \centering $1 +(4+7) = 1+11=12$
                             \vskip \centering $(1+4) + 7 =5+7= 12$ \vskip
\raggedright 
\item Propiedad asociativa (multiplicación): \vskip \centering $2\cdot(3\cdot5)=2\cdot15=30$
                             \vskip \centering $(2\cdot3)\cdot5=6\cdot5=30$
\vskip
\raggedright 
\item Propiedad distributiva (suma): \vskip \centering $2\cdot(3+5)=2\cdot8=16$
                             \vskip \centering $2\cdot(3+5)=2\cdot3+2\cdot5=6+10=16$ \vskip
\raggedright 
\item Propiedad distributiva (resta): \vskip \centering $3\cdot(8-3)=3\cdot5=15$
                             \vskip \centering $3\cdot(8-3)=3\cdot8-3\cdot3=24-9=15$

\end{itemize}
\textit{No te olvides de separar en términos:} Separar en términos nos sirve para saber qué cuenta hacer primero sin cometer errores:

\begin{equation}
    \notag
    5\cdot3+4 = \overbrace{5\cdot3} + 4 = 15+4=19 
\end{equation}



\raggedright
Ejercicios:


\begin{enumerate}
\renewcommand{\labelenumi}{{\theenumi})}
\item $10+5-2+1$
\item $15\cdot0+10\cdot3-25/5$
\item $(27+5)\cdot6-8\cdot(4+3\cdot2)$
\end{enumerate}