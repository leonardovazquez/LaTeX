Dado que hemos visto hasta aquí una amplia variedad de propiedades, intentemos resolver brevemente una ecuación un tanto difícil: 

\begin{equation}
    \notag
    \frac{2-3\cdot x}{4} = (-1+4)
\end{equation}

Resolvemos la resta de la derecha y luego lo que esta dividiendo a la izquierda pasa multiplicando hacia la derecha:

\begin{equation}
    \notag
    2-3\cdot x = 3\cdot 4 
\end{equation}

\begin{equation}
    \notag
    2-3\cdot x = 12
\end{equation}

El $2$ que suma pasa restando:

\begin{equation}
    \notag
    -3\cdot x = 12-2
\end{equation}

\begin{equation}
    \notag
    -3\cdot x = 10
\end{equation}

Finalmente pasamos dividiendo el $-3$:

\begin{equation}
    \notag
    x = \frac{-10}{3}
\end{equation}


Veamos otro ejemplo más difícil:

\begin{equation}
    \notag
    (x+5)^2=4
\end{equation}

La potencia la pasamos como raíz cuadrada:

\begin{equation}
    \notag
    x+5 = \sqrt{4} 
\end{equation}

La raíz cuadrada de 4 es 2, de esta forma despejamos fácilmente la incógnita:

\begin{equation}
    \notag
    x = 2-5 = -3 
\end{equation}

Ejercicios: Despejar $x$
\begin{enumerate}
\renewcommand{\labelenumi}{{\theenumi})}
\item $1/4 \cdot x -1/2\cdot 3/5 =2/3$
\item $(3/2 \cdot x -5)^2 = 4/9$
\item $3/2\cdot (x-3)+0,5 = 4/5 -1/3 \cdot x$
\end{enumerate}
