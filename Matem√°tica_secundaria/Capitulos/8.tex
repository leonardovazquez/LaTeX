Un sistema de ecuaciones de dos incógnitas consiste en dos ecuaciones. Lo que debemos hacer es despejar una de ellas en función de la otra de una de las dos ecuaciones. Luego, expresar en la segunda ecuación en función de una sola incógnita para luego hallar la segunda. ¡Qué trabalenguas! \\

Veamos un ejemplo:

\begin{equation}
    \notag
     \left\{ \begin{array}{lcc}
             x + y =3 
             \\2\cdot x + y = 4
             \end{array}
   \right.
\end{equation}

De la primer ecuación vamos a despejar $x$:

\begin{equation}
    \notag
    x = 3 - y
\end{equation}

Luego, vamos a la segunda ecuación y donde esté $x$ colocamos$(3+y)$

\begin{equation}
    \notag
    2\cdot  ( 3 - y ) +y =4
\end{equation}

Una vez que tenemos la ecuación de una sola incógnita, despejamos la $y$:

\begin{equation}
    \notag
    -\cdot y = 4- 6 =-2
\end{equation}

Por lo tanto $y=2$ y $x=1$.\\
\medskip

Ejercicios: Encontrar $x$ e $y$
\begin{enumerate}
\renewcommand{\labelenumi}{{\theenumi})}
\item 
    \begin{equation}
    \notag
     \left\{ \begin{array}{lcc}
             4\cdot x +y =-5 
             \\2\cdot x + y = 4
             \end{array}
   \right.
   \end{equation}
\item 
    \begin{equation}
    \notag
     \left\{ \begin{array}{lcc}
             -5\cdot x +2\cdot y =2
             \\3\cdot x-4\cdot  y = 5
             \end{array}
   \right.
     \end{equation}


\end{enumerate}

