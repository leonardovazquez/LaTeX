La manera de resolver una ecuación es despejar. Despejar significa dejar a la incógnita ($x$ o $y$ o cualquier letra) de un lado y pasar todo lo demás para el otro lado.\\
\medskip
Resolvamos esta ecuación:\\
\begin{equation}
    \notag
    2\cdot x + 5 = 13
\end{equation}

 
Acá tenemos que pasar primero el $5$ que está sumando. Lo pasamos para  el otro lado pero le cambiamos el signo:\\
\begin{equation}
    \notag
    2\cdot x = 13 - 5
\end{equation}


Luego tenemos que pasar el $2$ que está multiplicando a la incógnita $x$. Para eso lo pasamos hacia el otro lado dividiendo:\\

\begin{equation}
    \notag
    x = (13 - 5)/2
\end{equation}



De esta forma, resolvemos primero la operación entre paréntesis:\\

\begin{equation}
    \notag
    x = 8/2
\end{equation}


Por lo tanto nuestra incógnita está despejada y vale $x=4$.\\
\medskip
Las reglas básicas para pasar de términos son:

\begin{itemize}
    \item Lo que está sumando pasa restando:
    \begin{equation}
    \notag
       x+2=5 \longrightarrow x=5-2 
    \end{equation}

       
    \item Lo que está restando pasa sumando:
    \begin{equation}
    \notag
    x-3 = 9 \longrightarrow x=9+3 
    \end{equation}
    
    \item Lo que está multiplicando pasa dividiendo: 
    \begin{equation}
    \notag
     3\cdot x =4 \longrightarrow x=4/3 
    \end{equation}
   
    \item Lo que está dividiendo pasa multiplicando: 
    \begin{equation}
    \notag
       x/2=5 \longrightarrow x=5\cdot2 
    \end{equation}
  
\end{itemize}

Ejercicios:

\begin{enumerate}
\renewcommand{\labelenumi}{{\theenumi})}
\item $15=x+4$
\item $27+x=2$
\item $5\cdot x + 1 = 15$
\item $ 52-12 = x-19+20$
\item $2\cdot x +5\cdot ( 25-20) = 7\cdot7 -10$
\item $x\cdot6 -2 = -2$
\item $2\cdot(x+2)=-4$
\item $x+4\cdot(x+3)=0$
\item $3\cdot(2\cdot x+1)+x=5\cdot x$
\end{enumerate}